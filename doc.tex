\documentclass[a4paper,twoside=false,abstract=false,numbers=noenddot,
titlepage=false,headings=small,parskip=half,version=last]{scrartcl}
\usepackage[utf8]{inputenc}
\usepackage[T1]{fontenc}
\usepackage[english]{babel}
\usepackage[colorlinks=true, pdfstartview=FitV,
linkcolor=black, citecolor=black, urlcolor=blue]{hyperref}
\usepackage{verbatim}
\usepackage{graphicx}
\usepackage{multirow}

\usepackage{tikz}
\usetikzlibrary{matrix}

\usepackage{amsmath}
\usepackage{amsthm}
\usepackage{amssymb}
\usepackage{amsfonts}

\usepackage{float}

\usepackage{gensymb}

\usepackage{authblk}

\usepackage{helpers}


\title{Solid State Physics - IM2601}
\subtitle{Laboration 2}
    \author[1]{Fredrik Forsberg}
    \author[1]{Jim Holmström}
    \author[1]{Samuel Zackrisson}
    \affil[1]{Engineering Physics, Royal Institute of Technology}
    \affil[1]{\{fforsber, jimho, samuelz\}@kth.se}


\begin{document}
\maketitle
\thispagestyle{empty}

\section{Introduction}
The resistivity of a metal depends on its temperature. For many metals and alloys the resistivity falls to zero for temperatures below a critical temperature $T_c$, often in the liquid helium range. At this temperature the material undergoes a phase transition to a \emph{superconducting} state. Theses materials are called \emph{superconductors}.\\
There are two types of superconductors, type I and type II. They differ in how they are magnetized by an applied magnetic field. This difference is shown in figure \ref{fig:superconductortypes}.

\image{superconductortypes}{There are two types of superconductor, from \cite{Kittel}\label{fig:superconductortypes}}

\section{Experimental procedure}

\section{Measurements results}

\section{Discussion of the results}

\section{Conclusions}
Stuff happened, y0.

\begin{thebibliography}{1}
    \bibitem{Kittel}
        Charles Kittel,
        {\em Introduction to Solid State Physics 8th Edition},
        2005.
\end{thebibliography}


\end{document}
